\documentclass[a5paper, 10pt, twoside]{article} % тип документа

\usepackage{hyperref}
\usepackage[font=small,labelfont=bf]{caption}
\usepackage{fancyhdr}
\usepackage{import}

% Математика
\import{../headers/}{math.tex}

%  Русский язык
\import{../headers/}{russian.tex}

% Дефайны
\import{../headers/}{my_defs.tex}

%%%%%%%%%%%%%%%%%%%%%%
% МЕНЯЕМ АВТОРА ЗДЕСЬ !!!!!!!!!!!!!!!!
\newcommand{\SMB}{SOMEBODY}
% МЕНЯЕМ НОМЕР ЗДЕСЬ !!!!!!!!!!!!!!!!
\newcommand{\LNUM}{2.1}
%%%%%%%%%%%%%%%%%%%%%%
%\fancyhf{}
\renewcommand{\footrulewidth}{ .0em }
\fancyfoot[C]{\texttt{\textemdash~\thepage~\textemdash}}
\fancyhead[L]{Работа \LNUM \hfil}
\fancyhead[R]{\hfil \SMB, группа Б01-909}

\pagestyle{fancy}

\graphicspath{{./src/pics/}} % где лежат картинки

\counterwithin{figure}{section}

% Title Page
\title
{\hfill \break  \hfill \break
\hfill \break  \hfill \break
Лабораторная работа \LNUM.

Опыт Франка-Герца.
}
\author{\SMB, Б01-909}

%\setcounter{secnumdepth}{0}

\begin{document}

\maketitle



\thispagestyle{empty} % выключаем отображение номера для этой страницы

\newpage

\tableofcontents % Вывод содержания
\thispagestyle{plain}
\newpage


\paragraph{Цель работы:}
Методом электронного возбуждения измерить энергию первого уровня атома гелия в динамическом и статическом режимах.

\section{Теоретические сведения}
\import{./src/}{1.tex}

\newpage
\section{Экспериментальная установка}
\import{./src/}{2.tex}

\newpage
\section{Ход работы}
\import{./src/}{3.tex}

\newpage
\section{Вывод}

В ходе работы был воспроизведён опыт Франка-Герца, подтверждающий наличие
дискретных уровней возбуждения атомов. Вольт-амперная характеристика
трёхэлектродной вакуумной лампы была исследована двумя способами ~---~ динамическим и
статическим. Были экспериментально определены потенциалы возбуждения
атомов гелия (одноатомный газ, заполняющий лампу).

\begin{align*}
  E_{дин} = 16 \pm 4 &&
  E_{cтат} = 18 \pm 4
\end{align*}
Теоретическое значение:
$$
E_{теор} = 20.9
$$
\end{document}
