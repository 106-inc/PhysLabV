\renewcommand{\tg}{\mathop{\mathrm{tg}}\nolimits}
\renewcommand{\ctg}{\mathop{\mathrm{ctg}}\nolimits}
\renewcommand{\arctan}{\mathop{\mathrm{arctg}}\nolimits}
\newcommand{\divisible}{\mathop{\raisebox{-2pt}{\vdots}}}
\newcommand{\grad}{\mathop{\mathrm{grad}}\nolimits}
\newcommand{\diver}{\mathop{\mathrm{div}}\nolimits}
\newcommand{\rot}{\mathop{\mathrm{rot}}\nolimits}
\newcommand{\veci}{{\vec\imath}}                % i-орт
\newcommand{\vecj}{{\vec\jmath}}                % j-орт
\newcommand{\veck}{{\vec{k}}}                   % k-орт
\renewcommand{\phi}{\varphi}                    % Красивая фи
\newcommand{\der}[2]{\frac{d #1}{d #2}} % derivative
\newcommand{\dpart}[2]{\frac{\partial #1}{\partial #2}} % first partial der
\newcommand{\ddpart}[2]{\frac{\partial^2 #1}{\partial #2^2}} % вторая частная производная
\newcommand{\e}{\mathop{\mathrm e}\nolimits}    % Экспонента
\newcommand{\E}{\mathcal{E}}                    % ЭДС
\renewcommand{\epsilon}{\varepsilon}            % Красивый эпсилон
\newcommand{\degr}{\ensuremath{^\circ}}         % Градус
\newcommand{\const}{\ensuremath{\mathfrak{const}}} % постоянная
\renewcommand{\C}{\ensuremath{\mathfrak{C}}}    % Постоянная интегрирования
\newcommand{\Int}{\int\limits}          % Большой интеграл (можно поменять \int на \varint)
\newcommand{\IInt}{\iint\limits}          % Большой интеграл (можно поменять \int на \varint)
\newcommand{\Oint}{\oint\limits}                % Большой интеграл
\newcommand{\lvec}[1]{\overrightarrow{#1}}            % beauty vector arrow
\newcommand{\quot}[1]{<<#1>>}                   % russian quotes
\newcommand{\iu}{\imath}
\newcommand{\system}[1]{\left\lbrace\begin{array}{c} #1 \end{array} \right.}
\newcommand{\orsys}[1]{\left[\begin{array}{c} #1 \end{array} \right.}

\newcommand{\cvec}[1]{\left[\begin{array}{c} #1 \end{array} \right]}
\newcommand{\matr}[2]{\left(\begin{array}{#1} #2 \end{array} \right)}
\newcommand{\matrtwo}[1]{\matr{c c}{#1}}
\newcommand{\matrthree}[1]{\matr{c c c}{#1}}
\newcommand{\detmat}[2]{\left|\begin{array}{#1} #2 \end{array} \right|}
\newcommand{\dettwo}[1]{\detmat{c c}{#1}}
\newcommand{\detthree}[1]{\detmat{c c c}{#1}}
\newcommand{\gath}[1]{\begin{gathered} 	#1	\end{gathered}}