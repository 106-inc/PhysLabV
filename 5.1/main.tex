\documentclass[a5paper, 10pt, twoside]{article} % тип документа

\usepackage{hyperref}
\usepackage{fancyhdr}
\usepackage{import}

% Математика
\import{../headers/}{math.tex}

%  Русский язык
\import{../headers/}{russian.tex}

% Дефайны
\import{../headers/}{my_defs.tex}

%\fancyhf{}
\renewcommand{\footrulewidth}{ .0em }
\fancyfoot[C]{\texttt{\textemdash~\thepage~\textemdash}}
\fancyhead[L]{Работа 5.1 \hfil}
\fancyhead[R]{\hfil SOMEBEODY, группа Б01-909}

\pagestyle{fancy}

\graphicspath{{src/pics/}} % где лежат картинки

\counterwithin{figure}{section}

% Title Page
\title
{\hfill \break  \hfill \break
\hfill \break  \hfill \break
Лабораторная работа 5.1.

Измерение коэффициента ослабления потока $\gamma$-лучей в веществе и определение
их энергии}
\author{SOMEBEODY, Б01-909}

%\setcounter{secnumdepth}{0}

\begin{document}

\maketitle


\thispagestyle{empty} % выключаем отображение номера для этой страницы

\newpage

\tableofcontents % Вывод содержания
\thispagestyle{plain}
\newpage


\paragraph{Цель работы:}

С помощью сцинтилляционного счетчика измерить линейные коэффициенты ослабления
потока $\gamma$-лучей в свинце, железе и алюминии; по их величине определить
энергию $\gamma$-квантов.

\section{Теоретические сведения}

Гамма-лучи возникают при переходе возбужденных ядер из одного энергетического
состояния в другое, более низкое. Энергия $ \gamma $-квантов обычно заключена
между несколькими десятками килоэлектронвольт и несколькими миллионами
электрон-вольт. Гамма-кванты не несут электрического заряда, их масса равна
нулю. Проходя через вещество, пучок $ \gamma $-квантов постепенно ослабляется.
Ослабление происходит по экспоненциальному закону, который может быть записан в
двух эквивалентных нормах:

\begin{equation}\label{I(mu)}
I = I_0 e^{-\mu l}, \quad I_0 e^{-\mu 'm_1}
\end{equation}

В этих формулах $ I, I_0 $ --- интенсивности прошедшего и падающего излучений,
$ l $ --- длина пути, пройденного пучком $\gamma$-лучей, $ m_1 $ --- масса
пройденного вещества, приходящаяся на единицу площади, $ \mu $ и $ \mu' $ ---
константы, величина которых зависит от вещества, сквозь кото- рое проходят
$\gamma$-лучи. Длину пути $ l $ обычно выражают в сантиметрах, поэтому $ \mu $
имеет размерность см$ ^{-1} $; величину $ m_1 $ измеряют в г/см$ ^2 $, так что
размерность $ \mu' $ равна см$ ^2 $/г. Форма записи через массу является
предпочтительной, потому что $ \mu' $, в отличие от $ \mu $, не зависит от
плотности среды.

Ослабление потока $\gamma$-лучей, происходящее при прохождении среды, связано
с тремя эффектами: \textbf{фотоэлектрическим поглощением},
\textbf{комптоновским рассеянием} и с \textbf{генерацией электрон-позитронных
пар}. Рассмотрим эти эффекты.

  \subsection{Фотоэлектрическое поглощение.}
  \import{./src/}{1_1.tex}

  \subsection{Комптоновское рассеяние.}
  \import{./src/}{1_2.tex}

  \subsection{Образование пар}
  \import{./src/}{1_3.tex}

  \subsection{Полный коэффициент ослабления $\gamma$-лучей}
  \import{./src/}{1_4.tex}

\end{document}