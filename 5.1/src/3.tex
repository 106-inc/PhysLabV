Сначала измерим число частиц без поглотителя: $\langle N_0 \rangle \approx
175000$ частиц за $30$ секунд. Теперь перекроем колимматорный канал толстой
свинцовой пробкой. Результаты измерений за 30 секунд представлены в таблице
ниже.

\begin{table}[h!]
  \centering
  \caption{Измерение фона}
  \import{./src/tex_tbl/}{Back.tex}
\end{table}

Теперь проведем измерения, закрывая колимматорный канал поглотителями разной
толщины и из разных веществ. Результаты измерений в таблице ниже:

\begin{table}[h!]
  \centering
  \caption{Измерения для алюминия}
  \import{./src/tex_tbl/}{Al.tex}
\end{table}

\begin{table}
  \centering
  \caption{Измерения для железа}
  \import{./src/tex_tbl/}{Fe.tex}
\end{table}

\begin{table}
  \centering
  \caption{Измерения для свинца}
  \import{./src/tex_tbl/}{Pb.tex}
\end{table}

\begin{table}
  \centering
  \caption{Измерения для дерева}
  \import{./src/tex_tbl/}{Wood.tex}
\end{table}


