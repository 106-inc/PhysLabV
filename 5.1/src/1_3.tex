При энергиях $\gamma$-лучей, превышающих $2mc^2 = 1,02$ МэВ, становится
возможен процесс поглощения $\gamma$-лучей, связанный с образованием
электрон-позитронных пар. Рождение пар не может происходить в вакууме, оно
возникает в электрическом поле ядер. Вероятность этого процесса приблизительно
пропорциональна $Z^2$ и сложным образом зависит от энергии фотона. При
энергиях больше $2mc^22$ фотоэффект даже для самых тяжелых ядер уже не играет
практически никакой роли. Вероятность образования пар должна поэтому
сравниваться с вероятностью комптоновского рассеяния. При энергиях, с которыми
приходится иметь дело при изучении ядер, рождение пар существенно только в самых
тяжелых элементах. Так, даже для свинца вероятность рождения пар сравнивается с
вероятностью комптоновского эффекта только при энергии около $4,7$ МэВ.
