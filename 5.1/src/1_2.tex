Комптоновским рассеянием (или комптоновским эффектом) называется упругое
столкновение $\gamma$-кванта с электроном. При таком столкновении $\gamma$-квант
передает электрону часть своей энергии, величина которой определяется углом
рассеяния. В отличие от фотоэффекта, который может идти только на сильно
связанных электронах, комптоновское рассеяние происходит на свободных или
слабосвязанных электронах. Роль эффекта Комптона становится существенной только
тогда, когда энергия квантов становится много больше энергии связи электронов в
атоме (когда достаточно падает вероятность фотоэффекта). Атомные электроны в
этом случае можно считать практически свободными, что обычно и делается при
теоретическом анализе.

Вероятность комптон-эффекта сложным образом зависит от энергии $\gamma$-квантов.
В том случае, когда энергия $\gamma$-кванта много больше энергии покоя
электрона, формула сильно упрощается, и выражение для сечения комптон-эффекта
приобретает  вид:

\begin{equation}\label{sigma k}
\sigma_{\text{к}} = \pi r^2 \dfrac{mc^2}{\hbar\omega} \left( \ln{\dfrac{2\hbar\omega}{mc^2} + \dfrac{1}{2}} \right) 
\end{equation}

где $ r \simeq 2,8 \cdot 10^{-13} $ --- классический радиус электрона, $ m $ ---
его масса. Из формулы \eqref{sigma k} следует, что сечение комптон-эффекта с
ростом энергии фотонов падает далеко не так резко, как сечение фотоэффекта.
Сечение $\sigma_{\text{к}}$ относится к одному свободному электрону, в то время
как приведенное выше сечение фотоэффекта \eqref{sigma ph} рассчитано на атом.
Комптоновское рассеяние, отнесенное к атому, оказывается, естественно, в $Z$ раз
больше. 

Комптоновский коэффициент линейного ослабления $\mu_{\text{к}}$ связан с
сечением $\sigma_{\text{к}}$ формулой, аналогичной \eqref{mu ph}. Под $n$
следует в этом случае понимать плотность слабо связанных электронов, т. е.
практически полную плотность электронов в веществе. Отметим в заключение, что, в
отличие от фотоэффекта, эффект Комптона приводит не к поглощению
$\gamma$-квантов, а к их рассеянию и уменьшению их энергии.
