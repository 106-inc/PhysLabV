В нашей работе используется тиратрон ТГ3-01/1.3Б, заполненный инертным газом.
Схематическое изображение тиратрона и его конструкция приведены на рис.
(\ref{арыв}).

Электроны, эмитируемые катодом тиратрона, ускоряются напряжением $V$,
приложенным между катодом и ближайшей к нему сеткой. Затем электроны
рассеиваются на атомах инертного газа. Все сетки 1, 2, 3 соединены между собой и
имеют одинаковый потенциал, примерно равный потенциалу анода 6. Поэтому между
первой сеткой 1 и анодом практически нет поля. Рассеянные электроны отклоняются
в сторону и уходят на сетку, а оставшаяся часть электронов достигает анода и
создает анодный ток $I_a$. Таким образом, поток электронов $N(x)$ на расстоянии
$x$ от ускоряющей сетки (т.е. число электронов, проходящих через поперечное
сечение лампы в точке $x$ в единицу времени) уменьшается с ростом $x$ от
начального значения $N_0$ у катода (в точке $x = 0$) до некоторого значения
$N_a$ у анода (в точке $x = L$).

Рассмотрим теперь, какова должна быть реальная вольт-амперная характеристика
(ВАХ) тиратрона. Выделим в газе на расстоянии $x$ тонкий слой с площадью
поперечного сечения $S$ и толщиной $dx$. Этот слой содержит $\nu = n_a S dx$
атомов газа ($n_a$ -- концентрация атомов газа в лампе). Суммарная рассеивающая
поверхность этих атомов $\Delta = \mu \Delta_a$, где $\Delta_a$ -- площадь
поперечного сечения атома. Обозначим через $dN$ убыль потока электронов в
результате прохождения слоя $dx$; тогда $dN/N(x)$ есть доля электронов, которые
рассеялись, или вероятность рассеяния в слое. Для рассеяния электрона в слое
необходимо выполнение двух независимых событий -- электрон должен <<наткнуться>>
в слое на атом, и, кроме того, он должен на этом атоме рассеяться.
Следовательно, вероятность $dN/N(x)$ рассеяния электрона в слое равна
произведению двух вероятностей -- вероятности для электрона в слое $dx$
встретить атом газа (она равна $\Delta / S$ -- доли площади поперечного сечения
слоя, перекрываемого атомами) и вероятности рассeяния на атоме $\omega(V)$:

\begin{equation}
  - \frac{dN}{N(x)} = \frac{\Delta}{S} \cdot \omega(V) = n_a \Delta_a \omega(V) dx
\end{equation}

Интегрируя это соотношение от $0$ до $L$ и заменяя поток электронов на ток $I =
Ne$, получаем уравнение ВАХ:

\begin{equation} \label{eq::current}
  I_a = I_0 \e^{ - C \omega(V)}, \: C = L \cdot n_a \Delta_a
\end{equation}

где $I_0 = e N_0$ -- ток катода, $I_a = e N_a$ -- анодный ток.

Согласно классическим представлениям, сечение рассеяния электрона на атоме
должно падать монотонно с ростом $V$ (обратно пропорционально скорости
электрона, т.е. обратно пропорционально корню квадратному из его энергии), а
значит ВАХ будет монотонна возрастающей функцией, как это показано на рис.
(\ref{djflsk}). По квантовым соображения вероятность рассеяния электронов и
соответствующая ВАХ должны иметь вид, показанный на рис. (\ref{fjsdlkfsd}).

Согласно формуле \eqref{eq::current}, по измеренной ВАХ тиратрона можно
определить зависимость вероятности рассеяния электрона от его энергии из
соотношения

\begin{equation}
  \omega(V) = -\frac{1}{C} \cdot \ln \frac{I_a(V)}{I_0}
\end{equation}

Принципиальная схема установки для изучения эффекта Рамзауэра приведена на рис.
(\ref{fjsdlfks}). На лампу подается синусоидальное напряжение частоты $50$ Гц от
источника питания, $C$ -- стабилизированный блок накала катода; исследуемый
сигнал подается на электронный осциллограф; цифрами обозначены номера ножек
лампы.

Реально на экране осциллографа удается надежно наблюдать лиш один(первый, при $n
= 1$) минимум в сечении рассеяния электронов и следующий за ним максимум. Дело в
том, что уже при $n = 2$ напряженность поля столь велика, что с большой
вероятностью происходить ионизация атомов и возникает пробой тиратрона. Кроме
того, как показывает расчет, с ростом $n$ глубина минимума резко уменьшается,
что приводит к не столь ярко выраженному эффекту <<просветления>> газа.

Схема экспериментальной установки, изображенная на рис. (\ref{fjdslf}) в нашей
работе конструктивно осуществлена следующим образом. Лампа-тиратрон ТГЗ-01/1.3Б,
заполненная инертным газом, расположена непосредственно на корпусе блока
источника питания (БИП). Напряжение к электродам лампы подается от источников
питания, находящихся в корпусе прибора. Регулировка напряжения и выбор режима
работы установки производится при помощи ручек управления, выведенных на лицевую
панель БИП (рис. (\ref{fdsjlfks000})).
