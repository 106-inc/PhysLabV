\documentclass[a5paper, 10pt, twoside]{article} % тип документа

\usepackage{hyperref}
\usepackage[font=small,labelfont=bf]{caption}
\usepackage{fancyhdr}
\usepackage{import}

% Математика
\import{../headers/}{math.tex}

%  Русский язык
\import{../headers/}{russian.tex}

% Дефайны
\import{../headers/}{my_defs.tex}

%%%%%%%%%%%%%%%%%%%%%%
% МЕНЯЕМ АВТОРА ЗДЕСЬ !!!!!!!!!!!!!!!!
\newcommand{\SMB}{SOMEBODY}
% МЕНЯЕМ НОМЕР ЗДЕСЬ !!!!!!!!!!!!!!!!
\newcommand{\LNUM}{2.2}
%%%%%%%%%%%%%%%%%%%%%%
%\fancyhf{}
\renewcommand{\footrulewidth}{ .0em }
\fancyfoot[C]{\texttt{\textemdash~\thepage~\textemdash}}
\fancyhead[L]{Работа \LNUM \hfil}
\fancyhead[R]{\hfil \SMB, группа Б01-909}

\pagestyle{fancy}

\graphicspath{{./src/pics/}} % где лежат картинки

\counterwithin{figure}{section}

% Title Page
\title
{\hfill \break  \hfill \break
\hfill \break  \hfill \break
Лабораторная работа \LNUM.

Изучение спектров атомарного водорода и молекулярного йода.
}
\author{\SMB, Б01-909}

%\setcounter{secnumdepth}{0}

\begin{document}

\maketitle



\thispagestyle{empty} % выключаем отображение номера для этой страницы

\newpage

\tableofcontents % Вывод содержания
\thispagestyle{plain}
\newpage


\paragraph{Цель работы:}
исследовать сериальные закономерности в оптическом спектре водорода; спектр
поглощения паров йода в видимой области.

\section{Теоретические сведения}
\import{./src/}{1.tex}

\newpage
\section{Экспериментальная установка}
\import{./src/}{2.tex}

\newpage
\section{Ход работы}
\import{./src/}{3.tex}

\newpage
\section{Вывод}

В работе исследовались сериальные закономерности в оптическом спектре водорода и
спектр поглощения паров йода в видимой области.

Была построена градуировочная кривая по данным спектрам неона и ртути. Затем
получены длины волн линий $H_{\alpha}$, $H_{\beta}$ и $H_{\gamma}$ серии
Бальмера:

\[  H_{\alpha} = (650\pm 15)\ \text{нм} \]
\[  H_{\beta} = (486\pm 20)\ \text{нм} \]
\[  H_{\gamma} = (430\pm  25)\  \text{нм} \]


Вычислена постоянная Ридберга:

\[ \text{Ry}_p=(1.10\pm 0.05)\cdot 10^{-2} ~\text{нм}^{-1} \]

В пределах погрешности экспериментальное значение в пределах погрешности
совпадает с теоретическим:

\[ \text{Ry}_t=(1.097 \pm 0.001)\cdot 10^{-2} \ \text{нм}^{-1} \]


Получены длины волн, соответствующие некоторым электронно-колебательным
переходам из основного состояния в возбуждённое. Вычислены энергия
колебательного кванта возбуждённого состояния молекулы,

\[	h\nu_2=(1.0\pm 0.2)\cdot 10^{-2} \ \text{эВ} \]

Энергия электронного перехода:

\[ \Delta E=(2.0\pm 0.1) \ \text{эВ} \]

Энергии диссоциации молекулы в основном и в возбуждённом состояниях:

\[ D_1=(1.5\pm 0.1)\  \text{эВ} \]
\[ D_2=(0.42\pm 0.1) \ \text{эВ} \]


\end{document}
