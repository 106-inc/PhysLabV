Основными процессами взаимодействия гамма-излучения с веществом являются, как
было выше указано, фотоэффект, эффект Комптона и образование
электрон-позитронных пар. Каждый из этих процессов вносит свой вклад в
образование наблюдаемого спектра.

Фотоэффект -- процесс взаимодействия гамма-кванта с электроном, связанным с
атомом, при котором электрону передается вся энергия гаммакванта. При этом
электрону сообщается кинетическая энергия $T_e = E_{\gamma} -- I_i$, где
$E_{\gamma}$ -- энергия гамма-кванта, $I_i$ -- потенциал ионизации $i$-той
оболочки атома. Фотоэффект особенно существенен для тяжелых веществ, где он идет
с заметной вероятностью даже при высоких энергиях гамма-квантов. В легких
веществах фотоэффект становится заметен лишь при относительно небольших энергиях
гамма-квантов. Наряду с фотоэффектом, при котором вся энергия гамма-кванта
передается атомному электрону, взаимодействие гаммаизлучения со средой может
приводить к его рассеянию, т.е. отклонению от первоначального направления
распространения на некоторый угол.

Эффект Комптона -- упругое рассеяние фотона на свободном электроне,
сопровождающееся изменением длины волны фотона (реально этот процесс происходит
на слабо связанных с атомом внешних электронах). Максимальная энергия
образующихся комптоновских электронов соответствует рассеянию гамма-квантов на
$180^o$ и равна

\begin{equation}
  E_{\max} = \frac{\eta \omega}{1 + \frac{m c^2}{1 \eta \omega}}
\end{equation}
