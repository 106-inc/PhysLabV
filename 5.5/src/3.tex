Накопление спектра производим в течение 600 секунд. Спектры представим на рис.
1-6. Проведём измерение фона (см. рис. 7) и убедимся, что интенсивность его
спектра много меньше интенсивности спектра исследуемых образцов; также в спектре
фона нет пиков.

\begin{figure}
\begin{center}
  \includegraphics[width=0.8\linewidth]{Na.png}
  \caption{Спектр ${}^{22}{\text{Na}}$}
  \includegraphics[width=0.8\linewidth]{Co.png}
  \caption{Спектр ${}^{60}{\text{Co}}$}
\end{center}
\end{figure}

\begin{figure}
\begin{center}
  \includegraphics[width=0.8\linewidth]{Cs.png}
  \caption{Спектр ${}^{137}{\text{Cs}}$}
  \includegraphics[width=0.8\linewidth]{Am.png}
  \caption{Спектр ${}^{241}{\text{Am}}$}
\end{center}
\end{figure}

\begin{figure}
\begin{center}
  \includegraphics[width=0.8\linewidth]{Eu.png}
  \caption{Спектр ${}^{152}{\text{Eu}}$}
\end{center}
\end{figure}

Используя известные значения пиков в спектрах натрия и цезия, построим
калибровочный график соответствия номера канала определённому значению энергии
(рис. 8).

\begin{figure}[h!]
  \centering
  \includegraphics[width=0.8\linewidth]{channel_from_energy.png}
  \caption{Калибровочный график}
\end{figure}

Получаем уравнение для перехода от номера канала к значению энергии в кэВ:
\begin{center}
    $E = (0.718 \pm 0.004)N_i - (17 \pm 5)$
\end{center}

Используя калибровочный график, определим для всех остальных источников значения
энергии пиков полного поглощения $E_i$ , их ширины на половине высоты $\Delta
E_i$ и энергетическое разрешение $R_i$ . Результаты занесём в таблицу \ref{tab:my_label}.

\begin{table}[h]
\begin{center}
  \caption{Пики полного поглощения различных образцов}
  \label{tab:my_label}
  \import{./src/}{picks.tex}
\end{center}
\end{table}

По графикам определим энергию характеристического излучения свинца, служащего
защитой спектрометра от внешнего излучения. На всех спектрах, кроме спектра
неизвестного образца, снятого вне установки, в той или иной степени выражена
спектральная линия, соответствующая энергии $75$ кэВ. Эта энергия и есть энергия
характеристического излучения свинца.

Проверим зависимость (6). Для этого построим график зависимости $R^2 = f(1/E)$
(рис. 10). Наблюдается линейная зависимость. Из-за неточностей в определении
полуширины пиков точки не лежат на одной прямой.

Определим энергии края комптоновского поглощения для образцов $^{22}$Na,
$^{137}$Cs, $^{60}$Co, сравним их с соответствующими справочными значениями.

\begin{table}
\begin{center}
  \caption{Комптоновские спектры}
  \import{./src/}{comp.tex}
\end{center}
\end{table}

В спектрах, где наблюдаются пики обратного рассеяния, определим энергии этих
пиков и сравним измеренные значения с определёнными по формуле (2)

\begin{equation}
  E_{b_s} = \frac{E_{\gamma}}{1 + \frac{2E_{\gamma}}{m_e c^2}}
\end{equation}

\begin{table}
  \begin{center}
  \begin{tabular}{| c | c | c |}
  \hline
  Образец & $E_{C_{\text{exp}}}$, МэВ & $E_{C_{\text{th}}}$, МэВ \\
  \hline
  ${}^{60}$Co ($E = 1.171$ МэВ) & $0,228$ & $0.209$ \\
  \hline
  ${}^{60}$Co ($E = 1.333$ МэВ) & $0,228$ & $0.214$ \\
  \hline
  ${}^{137}$Cs ($E = 0.662$ МэВ) & $0,198$ & $0.184$ \\
  \hline
\end{tabular}
\end{center}
\end{table}

Эти значения практически совпадают. Пики обратного рассеяния в спектре кобальта,
отвечающие разным пикам полного поглощения, на графике неразрешимы (виден один
широкий пик).

\begin{table}[h!]
  \centering
  \caption{Пики прямого поглощения}
  \begin{tabular}{| c | c | c | c | c | c | c |}
\hline
Элемент & $N_i$ & $\sigma_{N_i}$ & $E_i$, кэВ & $\sigma_{E_i}$, кэВ & $R_i$ & $\sigma_{R_i}$\\
\hline
$^{22}\text{Na}$ & $740$ & $27$ & $742$ & $31$ & $0.042$ & $0.002$\\
\hline
$^{22}\text{Na}$ & $1800$ & $42$ & $1840$ & $54$ & $0.0298$ & $0.0009$\\
\hline
$^{137}\text{Cs}$ & $940$ & $30$ & $949$ & $36$ & $0.038$ & $0.001$\\
\hline
$^{60}\text{Co}$ & $1657$ & $40$ & $1692$ & $51$ & $0.0306$ & $0.0009$\\
\hline
$^{60}\text{Co}$ & $1865$ & $43$ & $1907$ & $56$ & $0.0294$ & $0.0009$\\
\hline
$^{241}\text{Am}$ & $132$ & $11$ & $113$ & $13$ & $0.11$ & $0.01$\\
\hline
$^{152}\text{Eu}$ & $99$ & $9$ & $79$ & $11$ & $0.14$ & $0.02$\\
\hline
$^{152}\text{Eu}$ & $216$ & $14$ & $200$ & $16$ & $0.081$ & $0.007$\\
\hline
$^{152}\text{Eu}$ & $378$ & $19$ & $368$ & $21$ & $0.059$ & $0.003$\\
\hline
\end{tabular}

\end{table}
