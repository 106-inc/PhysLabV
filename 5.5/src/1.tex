Основная задача спектрометрических измерений заключается в определении энергии,
интенсивности дискретных гамма-линий от различных гамма-источников и их
идентификации. В настоящее время используются различные типы гамма-детекторов:
полупроводниковые, сцинтилляционные, пластиковые, жидкостные, газовые и т.д. Они
существенно отличаются как по своим спектрометрическим свойствам, так по
эксплуатационным характеристикам и по технологии и стоимости изготовления.

В данной работе исследуются сцинтилляционные гамма-спектрометры на основе
неорганического кристалла NaI(Tl) и органической сцинтиллирующей пластмассы.

При прохождении гамма-квантов через материальную среду образуются электроны,
возникающие за счет фотоэффекта, комптоновского рассеяния и рождения
электрон-позитронных пар. Теоретическое описание взаимодействия гамма-квантов с
веществом для этих эффектов подробно разобрано в главе V учебного пособия:
Игошин Ф.Ф., Самарский Ю.А., Ципенюк Ю.М. Лабораторный практикум по общей
физике. Квантовая физика – М.:Физматкнига, 2012. Образующиеся при этих процессах
электроны испытывают большое количество неупругих соударений с молекулами и
атомами среды. Неупругие соударения могут сопровождаться как ионизацией, так и
возбуждением молекул или атомов среды. В промежуточных же стадиях (при переходах
возбужденных молекул или атомов в основное состояние, при рекомбинации
электрических зарядов и т.п.) в веществе возникают кванты света различных длин
волн, присущих данному веществу. Процесс возникновения световых вспышек в
различных сцинтилляторах подробно разобран в Приложении II (пункты 2 и 3).

Вообще говоря, возникающее излучение должно сильно поглощаться в сцинтилляторе,
так как его энергия в точности равна энергии возбуждения атомов среды. Чтобы
избежать этого явления, в кристаллы сцинтиллятора вводят небольшие добавки
других атомов (в случае кристалла NaI это атомы таллия). При этом спектр
поглощения сдвигается относительно спектра испускания в сторону меньших длин
волн, и увеличивается вероятность выхода из вещества хотя бы некоторой части
квантов света, отвечающих длинноволновому краю спектра испускания. В этом случае
прохождение ионизирующей частицы через вещество будет сопровождаться световой
вспышкой, которая и может быть использована для регистрации частицы.

Широкое применение сцинтилляционный метод исследования излучений нашел после
того, как были изобретены и усовершенствованы фотоэлектронные умножители (ФЭУ),
позволяющие регистрировать весьма малые по длительности и очень слабые по
интенсивности вспышки света. Таким образом, современный сцинтилляционный счетчик
состоит, в принципе, из сцинтиллятора — вещества, способного испускать видимое
или ультрафиолетовое излучение, возникающее под действием заряженных частиц, и
фотоэлектронного умножителя, в котором энергия этих световых вспышек через
посредство фотоэффекта преобразуется в импульсы электрического тока.

\subsection{Процессы взаимодействия гамма-излучения с веществом.}
\import{./}{1_1.tex}

\subsection{Процесс образования электрон-позитронных пар.}
\import{./}{1_2.tex}

\subsection{Энергетическое разрешение спектрометра.}
\import{./}{1_3.tex}