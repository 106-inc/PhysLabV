При достаточно высокой энергии гамма-кванта наряду с фотоэффектом и эффектом
Комптона может происходить третий вид взаимодействия гамма-квантов с веществом –
образование электрон-позитронных пар. Процесс образования пар не может
происходить в пустоте, так как в этом случае не выполняются законы сохранения
энергии и импульса. В присутствии ядра или электрона процесс образования пары
гамма-квантом возможен, так как можно распределить энергию и импульс
гамма-кванта между тремя частицами без противоречия с законами сохранения. При
этом если процесс образования пары идет в кулоновском поле ядра или протона, то
энергия образующегося ядра отдачи оказывается весьма малой, так что пороговая
энергия гамма-кванта E0, необходимая для образования пары, практически совпадает
с удвоенной энергией покоя электрона Е0  2mc 2=1,022 МэВ. Появившийся в
результате процесса образования пар электрон теряет свою энергию на ионизацию
среды. Таким образом, вся энергия электрона остается в детекторе. Позитрон будет
двигаться до тех пор, пока практически не остановится, а затем аннигилирует с
электроном среды, в результате чего появятся два гамма-кванта. Т.е.,
кинетическая энергия позитрона также останется в детекторе. Далее возможны три
варианта развития событий: а) оба родившихся гамма-кванта не вылетают из
детектора, и тогда вся энергия первичного гамма-кванта останется в детекторе, а
в спектре появится пик с Е = Е; б) один из родившихся гамма-квантов покидает
детектор, и в спектре появляется пик, соответствующий энергии Е = ЕE0, где
Е0= mc 2=511 кэВ; в) оба родившихся гамма-кванта покидают детектор, и в спектре
появляется пик, соответствующий энергии Е = ЕE0, где 2Е0=2mc 2=1022 кэВ.
Таким образом, любой спектр, получаемый с помощью гаммаспектрометра, описывается
несколькими компонентами, каждая из которых связана с определенным физическим
процессом. Как описано выше, основными физическими процессами взаимодействия
гамма-квантов с веществом являются фотоэффект, эффект Комптона и образование
электронпозитронных пар, и каждый из них вносит свой вклад в образование
спектра. Помимо этих процессов, добавляются экспонента, связанная с наличием
фона, пик характеристического излучения, возникающий при взаимодействии
гамма-квантов с окружающим веществом, а также пик обратного рассеяния,
образующийся при энергии квантов Еγ >> mc 2 /2 в результате рассеяния
гамма-квантов на большие углы на материалах конструктивных элементов детектора и
защиты. Положение пика обратного рассеяния определяется по формуле: где Е –
энергия фотопика.  , (2) http://mephi.ru/content/news/1810/31483/ На рис.1
приведен в качестве примера спектр, полученный при регистрации сцинтилляционным
гамма-спектрометром гамма-квантов, излучаемых радиоактивными 60Co.

При бета-распаде 60Со образуется радиоактивный 60Ni в возбужденных состояниях с
энергиями 2,505 МэВ либо 1,332 МэВ. Распад высокоэнергетичного состояния
происходит преимущественно на первый возбужденный уровень и при этом испускается
гамма-квант с энергией 1,173 МэВ. На этом рисунке видны два фотопика,
соответствующие этим квантам. Кроме того, наблюдается непрерывное комптоновское
излучение, пик обратного рассеяния и характеристическое излучение из свинца,
служащего защитой детектора от космического излучения. На рис. 2 приведен еще
один спектр, зарегистрированный от радиоактивного изотопа 137Cs, который
является источником гамма-квантов с энергией 661,7 кэВ. На этом спектре
сплошными линиями показаны пик полного поглощения, обусловленный фотоэффектом,
пик характеристического излучения, пик обратного рассеяния, и экспоненциальный
фон. На рис. 3 приведен спектр, полученный от радиоактивного изотопа 22Na.
Источник 22Na кроме -излучения испускает позитроны, которые при аннигиляции с
электронами дают в спектре резкий аннигиляционный пик, соответствующий энергии
511 кэВ. В спектре этого источника есть также гамма-кванты с энергией 1,275 МэВ.

