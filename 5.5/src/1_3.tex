Даже при поглощении частиц с одинаковой энергией амплитуда импульса на выходе
фотоприёмника сцинтилляционного детектора меняется от события к событию. Это
связано: 1) со статистическим характером процессов сбора фотонов на
фотоприёмнике и последующего усиления, 2) с различной вероятностью доставки
фотона к фотоприёмнику из разных точек сцинтиллятора, 3) В результате с
разбросомв набранном высвечиваемого спектречисла линияфотонов (которая. для
идеального детектора представляла бы дельта-функцию) оказывается размытой, её
часто описывают гауссианом. Энергетическим разрешением спектрометра называется
величина

где – ширина пика полного поглощения, измеренная на половине высоты, i i E E –
энергия регистрируемого  - излучения. Значение Ei пропорционально среднему
числу фотонов i n на выходе ФЭУ, т.е.:

Поскольку энергетическое разрешение зависит от энергии, его следует указывать
для конкретной энергии. Чаще всего разрешение указывают для энергии гамма-линии
137Cs (661,7 кэВ). Принципиальная блок-схема гамма-спектрометра, изучаемого в
данной работе, показана на рис. 4. На этом рисунке: 1 – сцинтиллятор, 2 – ФЭУ, 3
– предусилитель импульсов, 4 – высоковольтный блок питания для ФЭУ, 5 – блок
преобразования аналоговых импульсов с ФЭУ в цифровой код (АЦП), 6 – компьютер
для сбора данных, их обработки и хранения. ФЭУ со сцинтиллятором и блоком
питания установлены на отдельной подставке. В нашей работе на разных установках
(стендах) в качестве сцинтиллятора используются кристаллы NaI(Tl) с размерами 
4550 мм и  2025.
