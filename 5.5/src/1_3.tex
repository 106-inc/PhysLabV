Даже при поглощении частиц с одинаковой энергией амплитуда импульса на выходе
фотоприёмника сцинтилляционного детектора меняется от события к событию. Это
связано: 1) со статистическим характером процессов сбора фотонов на
фотоприёмнике и последующего усиления, 2) с различной вероятностью доставки
фотона к фотоприёмнику из разных точек сцинтиллятора, 3) с разбросом
высвечиваемого числа  фотонов.

В результате в набранном спектре линия (которая для идеального детектора
представляла бы дельта-функцию) оказывается размытой, её часто описывают
гауссианом. Энергетическим разрешением спектрометра называется величина


\begin{equation}
  R_i = \frac{\Delta E_i}{E_i}
\end{equation}

где $\Delta E_i$ – ширина пика полного поглощения, измеренная на половине
высоты, $E_i$ – энергия регистрируемого  - излучения. Значение $E_i$
пропорционально среднему числу фотонов $\overline{n_i}$ на выходе ФЭУ, т.е.:

\begin{equation}
  E_i = \alpha \overline{n_i}
\end{equation}

Полуширина пика полного поглощения $\Delta E_i$ пропорциональна
среднеквадратичной флуктуации $\overline{\Delta n_i}$. Т.к. $n_i$ является
дискретной случайной величиной, которая распределена по закону Пуассона, то
$\overline{\Delta n_i} = \sqrt{\overline{n_i}}$ и поэтому

\begin{equation}
  \Delta E_i = \alpha \overline{\Delta n_i} = \alpha \sqrt{\overline{n_i}}
\end{equation}

Из (4), (5) получаем, что

\begin{equation}
  R_i = \frac{\Delta E_i}{E_i} = \frac{\const}{\sqrt{E_i}}
\end{equation}

Поскольку энергетическое разрешение зависит от энергии, его следует указывать
для конкретной энергии. Чаще всего разрешение указывают для энергии гамма-линии
${}^{137}{\text{Cs}}$ ($661,7$ кэВ).
