\documentclass[a5paper, 10pt, twoside]{article} % тип документа

\usepackage{hyperref}
\usepackage[font=small,labelfont=bf]{caption}
\usepackage{fancyhdr}
\usepackage{import}

% Математика
\import{../headers/}{math.tex}

%  Русский язык
\import{../headers/}{russian.tex}

% Дефайны
\import{../headers/}{my_defs.tex}

%\fancyhf{}
\renewcommand{\footrulewidth}{ .0em }
\fancyfoot[C]{\texttt{\textemdash~\thepage~\textemdash}}
\fancyhead[L]{Работа 5.5 \hfil}
\fancyhead[R]{\hfil SOMEBEODY, группа Б01-909}

\pagestyle{fancy}

\graphicspath{{./src/pics/}} % где лежат картинки

\counterwithin{figure}{section}

% Title Page
\title
{\hfill \break  \hfill \break
\hfill \break  \hfill \break
Лабораторная работа 5.5.

Компьютерная сцинтилляционная $\gamma$-спектрометрия}
\author{SOMEBEODY, Б01-909}

%\setcounter{secnumdepth}{0}

\begin{document}

\maketitle



\thispagestyle{empty} % выключаем отображение номера для этой страницы

\newpage

\tableofcontents % Вывод содержания
\thispagestyle{plain}
\newpage


\paragraph{Цель работы:}

Изучить спектр гамма-излучений для образцов ${}^{22}{\text{Na}}$,
${}^{127}{\text{Cs}}$, ${}^{60}{\text{Co}}$, ${}^{241}{\text{Am}}$ и
${}^{152}{\text{Eu}}$, найти для них пики полного поглощения и обратного
рассеяния.

\section{Теоретические сведения}
\import{./src/}{1.tex}

\newpage
\section{Экспериментальная установка}
\import{./src/}{2.tex}

\newpage
\section{Ход работы}
\import{./src/}{3.tex}

\end{document}
