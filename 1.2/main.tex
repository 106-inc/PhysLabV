\documentclass[a5paper, 10pt, twoside]{article} % тип документа

\usepackage{hyperref}
\usepackage[font=small,labelfont=bf]{caption}
\usepackage{fancyhdr}
\usepackage{import}

% Математика
\import{../headers/}{math.tex}

%  Русский язык
\import{../headers/}{russian.tex}

% Дефайны
\import{../headers/}{my_defs.tex}

%\fancyhf{}
\renewcommand{\footrulewidth}{ .0em }
\fancyfoot[C]{\texttt{\textemdash~\thepage~\textemdash}}
\fancyhead[L]{Работа 1.2 \hfil}
\fancyhead[R]{\hfil SOMEBEODY, группа Б01-909}

\pagestyle{fancy}

\graphicspath{{./src/pics/}} % где лежат картинки

\counterwithin{figure}{section}

% Title Page
\title
{\hfill \break  \hfill \break
\hfill \break  \hfill \break
Лабораторная работа 1.2.

Исследование эффекта Комптона.}
\author{SOMEBEODY, Б01-909}

%\setcounter{secnumdepth}{0}

\begin{document}

\maketitle



\thispagestyle{empty} % выключаем отображение номера для этой страницы

\newpage

\tableofcontents % Вывод содержания
\thispagestyle{plain}
\newpage


\paragraph{Цель работы:}
С помощью сцинтилляционного счётчика спектрометра исследуется энергетический
спектр $\gamma$-квантов, рассеянных на графите. Определяется энергия рассеянных
$\gamma$-квантов в зависимости от угла рассеяния, а также энергия покоя частиц,
на которых происходит комптоновское рассеяние.

\section{Теоретические сведения}
\import{./src/}{1.tex}

\newpage
\section{Экспериментальная установка}
\import{./src/}{2.tex}

\newpage
\section{Ход работы}
\import{./src/}{3.tex}

\newpage
\section{Вывод}

Итак, в настоящей лабораторной работе нами была проведена проверка соотношения
(\ref{Kompton}). Экспериментально установлено, что $\gamma$-кванты действительно
испытывают упругое рассеяние на свободных частицах.

Обратим наше внимание на то, что с увеличением угла $\theta$ погрешность
измерения номера канала $\sigma_N$ увеличивается, что связано со смещением
фотопика в сторону сплошного распределения, обязанного комптоновскому
рассеянию. При $\theta = 110^\circ$ уже было невозможно увидеть пик полного
поглощения.

На основании таблицы \ref{N} можно определить энергию покоя частиц, на которых
происходит комптоновское рассеивание. Путем несложных преобразований формула
(\ref{1-cos}) принимает вид:
\begin{equation*}
  mc^2 = E(0) \frac{N_\text{наил}(90)}{N_\text{наил}(0)-N_\text{наил}(90)},
\end{equation*}
где $E(0)$ -- энергия $\gamma$-лучей, испускаемых источником (в нашем случае
$^{137}$Cs), то есть 662 кэВ. Имеем:
\[
  \boxed{mc^2 = 430 \pm 20 \ \text{кэВ}}.
\]
Видно, что результат на 16\% меньше 511 кэВ -- энергии покоя электрона.
Почему? Погрешность должна быть больше...

Отметим, что колебания напряжения на ФЭУ № 4012 практически не вносят
погрешности в измерения, ведь в работе было использовано напряжение $U = 1,6$
кэВ, при котором счетная характеристика выходит на константу, как видно из
рисунка~\ref{fig:U}.


\end{document}
