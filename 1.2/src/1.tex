Рассеяние $\gamma$ -лучей в веществе относится к числу явлений, в которых
особенно ясно проявляется двойственная природа излучения. Волновая теория,
хорошо объясняющая рассеяние длинноволнового иЗлучения, испытывает трудности при
описании рассеяния рентгеновских и $\gamma$ -лучей. Эта теория, в частности, не
может объяснить, почему в составе рассеянного излучения, измеренного Комптоном,
кроме исходной волны с частотой $\omega_{0}$ появляется дополнительная
длинноволновая компонента, отсутствующая в спектре первичного излучения.

Появление этой компоненты легко объяснимо, если считать, что $\gamma$-излучение
представляет собой поток квантов (фотонов), имеющих энергию $\hbar \omega$ и
импульс $p=\hbar \omega / c .$ Эффект Комптона - увеличение длины волны
рассеянного излучения по сравнению с падающим - интерпретируется как результат
упругого соударения двух частиц: $\gamma$ -кванта (фотона) и свободного
электрона.

Рассмотрим элементарную теорию эффекта Комптона. Пусть электрон до соударения
покоился (его энергия равна энергии покоя $m c^{2}$ ), a $\gamma$ -квант имел
начальную энергию $\hbar \omega_{0}$ и импульс $\hbar \omega_{0} / c .$ После
соударения электрон приобретает энергию $\gamma m c^{2}$ и импульс $\gamma m v,$
где $\gamma=$ $=\left(1-\beta^{2}\right)^{-1 / 2}, \beta=v / c,$ a $\gamma$
-квант рассеивается на некоторый угол $\theta$ по отношению к первоначальному
направлению движения. Энергия и импульс $\gamma$ -кванта становятся
соответственно равным и $\hbar \omega_{1}$ и $\hbar \omega_{1} / c($ рис. 1$)$.
Запишем для рассматриваемого процесса законы сохранения энергии и импульса:
$$
  \begin{array}{c}
    m c^{2}+\hbar \omega_{0}=\gamma m c^{2}+\hbar \omega_{1}                                  \\
    \\
    \frac{\hbar \omega_{0}}{c}=\gamma m v \cos \varphi+\frac{\hbar \omega_{1}}{c} \cos \theta \\
    \\
    \gamma m v \sin \varphi=\frac{\hbar \omega_{1}}{c} \sin \theta
  \end{array}
$$
Решая совместно эти уравнения и переходя от частот $\omega_{0}$ и $\omega_{1}$ к
длинам волн $\lambda_{0}$ и $\lambda_{1},$ нетрудно получить, что изменение
длины волны рассеянного излучения равно
\begin{equation}\label{DeltaLambda}
  \Delta \lambda=\lambda_{1}-\lambda_{0}=\frac{h}{m c}(1-\cos \theta)=\Lambda_{\mathrm{K}}(1-\cos \theta)
\end{equation}
где $\lambda_{0}$ и $\lambda_{1}$ - длины волн $\gamma$ -кванта до и после
рассеяния, а величина
$$
  \Lambda_{\mathrm{K}}=\frac{h}{m c}=2,42 \cdot 10^{-10} \mathrm{cm}
$$

называется комптоновской длиной волны электрона. Из формулы \eqref{DeltaLambda}
следует, что комптоновское смещение не зависит ни от длины волны первичного
излучения, ни от рода вещества, в котором наблюдается рассеяние. В приведенном
выводе электрон в атоме считается свободным. Для $\gamma$-квантов с энергией в
несколько десятков, а тем более сотен килоэлектрон-вольт, связь электронов в
атому, действительно, мало существенна, так как энергия, так как энергия их
связи в легких атомах не превосходит нескольких килоэлектрон-вольт, а для
большинства электронов еще меньше.

Основной целью данной работы является проверка соотношения \eqref{DeltaLambda}.
Применительно к условиям нашего опыта формулу \eqref{DeltaLambda} следует
преобразовать от длин волн к энергии $\gamma$ -квантов. Как нетрудно показать,
соответствующее выражение имеет вид
\begin{equation}\label{Eps}
  \frac{1}{\varepsilon(\theta)}-\frac{1}{\varepsilon_{0}}=1-\cos \theta
\end{equation}
Здесь $\varepsilon_{0}=E_{0} /\left(m c^{2}\right)-$ выраженная в единицах
$mc^{2}$ энергия $\gamma$-квантов, падающих на рассеиватель,
$\varepsilon(\theta)$ - выраженная в тех же единицах энергия квантов, испытавших
комптоновское рассеяние на угол $\theta$, $m$ -- масса электрона.
